\documentclass[11pt]{article}

\usepackage{amsmath}    % Expanded math
\usepackage{amssymb}    % Expanded math symbols
\usepackage{graphicx}   % For images
\usepackage[version=3]{mhchem}  % Nuclide formatting
\usepackage{hyperref}           % Hyperlinks to URLs

\usepackage{geometry}
\geometry{left=2.5cm,right=2.5cm,top=2.5cm,bottom=2.5cm}

% Metadata
\title{%
       NE 204: Advanced Concepts in Radiation Detection and \\ Measurement \\
       \Large\bf Experiment 0: Energy Calibration with Reproducible Workflows}
       \author{Ross Barnowski \\ {\tt rossbar@berkeley.edu}}
\date{Revised \today}

%==============================================================================
\begin{document}
\maketitle

\section*{Purpose}

The purpose of this ``experiment'' is to provide the opportunity to practice 
generating lab reports using the reproducible workflow required for this
course. 
Energy calibration of a HPGe detector (a ubiquitous task in radiation detection
laboratories) will be used as an example for practicing the workflow.
You must implement an approach to downloading data (see next section), 
extract the spectral information, and use it to develop a model for energy 
calibration. 
The energy calibration will then be evaluated using (at least) one of the 
datasets that was not used in the calibration procedure.
These results, along with any relevant discussion, are to be included in the
lab writeup.

\section*{Approach}

This is not a real lab in the sense that you will not be collecting data 
yourself.
The purpose is to practice writing lab reports, so data for the task of
energy calibration will be provided to you.
Of course, students are welcome to collect their own callibration spectra
from any of the available HPGe detectors, but the emphasis must be on 
fulfilling the software-related tasks discussed below.

\subsubsection*{Data}

The data used for this experiment has already been collected.
It consists of pulse height spectra taken with a coaxial HPGe detector using 
5 different radionuclide calibration sources: \ce{^{241}Am}, \ce{^{133}Ba},
\ce{^{137}Cs}, \ce{^{60}Co}, and \ce{^{152}Eu}.
The MCA with which the data was collected had 13-bit resolution, yielding 8192
channels.
The data for all five pulse height spectra are stored in a single text file in
ASCII format. 
The first row of the file is a header indicating which radionuclide the 
data in the columns correspond to, followed by 8192 rows indicating the number
of counts measured in each MCA channel.
The link to the data file and the corresponding md5 checksum are available at:
\begin{itemize}
  \item {\bf Data:} \url{https://www.dropbox.com/s/k692avun0144n90/lab0_spectral_data.txt?dl=0}
  \item {\bf Checksum:} \url{https://www.dropbox.com/s/6jquiryg6jskii0/lab0_spectral_data.md5?dl=0}
\end{itemize}

{\bf Required}
\begin{itemize}
  \item Implement the {\tt data} command in the {\tt Makefile} such that the
        data at the above link will be downloaded when the user types
        {\tt make data}. 
        \textit{Hint: take a look at {\tt wget} or {\tt curl} - command line tools 
                used for downloading files}.
\end{itemize}

{\bf Optional}
\begin{itemize}
  \item Implement a {\tt validate} command in the {\tt Makefile} that will
        compute the md5 checksum for the downloaded data, and compare it
        against the provided checksum to verify that the data has not
        been corrupted.
        \textit{Hint: the {\tt md5sum} command with the {\tt -c} flag might
                prove useful.}
\end{itemize}

\subsubsection*{Analysis}

Once the calibration data has been successfully downloaded, the energy 
calibration can be done.
The first step will be to load the calibration data into IPython 
\textit{Hint: check out {\tt np.loadtxt}}.
The next task is to determine the centroids of the peaks in the pulse height
spectrum that correspond to known gamma-ray energies
\textit{Hint: your experience with {\tt scipy.optimize.curve\_fit} from the first
        lab session should prove useful here}.
Finally, once the parameters of the energy calibration model have been 
determined, the quality of the calibration should be evaluated by applying it
to one of the pulse height spectra that was not used in the calibration 
procedure.
Compare the calibrated peak locations to their ``true'' values as specified in
the nuclear data literature.
To minimize the amount of effort required for the calibration itself, the
following procedure is proposed:

{\bf Required}
\begin{itemize}
  \item Conduct a two-point, linear energy calibration using the 59.541 keV
        peak from \ce{^{241}Am} and the 661.657 keV peak from \ce{^{137}Cs}.
  \item Apply the linear calibration model to the \ce{^{133}Ba} dataset and 
        quantify the difference between the calibrated peak locations and the 
        expected peak locations based on the literature.
\end{itemize}

{\bf Optional}
\begin{itemize}
  \item Use more than two points to determine the parameters of the linear
        calibration model via regression methods.
  \item Use data from the other sources (e.g. \ce{^{152}Eu}) to evaluate the
        energy calibration.
\end{itemize}

This analysis should result in the production of \textit{at least} one figure
that will be included in the report.
The generation of these figures should be automated by implementing the 
{\tt analysis} command in the {\tt Makefile}.

\subsubsection*{Testing}

{\bf Required}
\begin{itemize}
  \item Design and implement \textit{at least one} unit test for your data
        analysis code.
\end{itemize}

\section*{Report}

Make sure to update the project {\tt README.md} with instructions for how to 
download the data, replicate your analysis, run your software test suite, and
generate the document.
The report itself should contain the following:

\begin{enumerate}
  \item Describe the motivations and goals of this experiment. Why is energy
        calibration of detectors important?
  \item Discuss the approach to the energy calibration (N.B. there is no
        ``experimental setup'' in this case).
  \item Present the results of your energy calibration and include any
        relevant discussion.
\end{enumerate}

%\section*{Submission}
%
%All lab reports will be submitted via github.
%You can submit your lab report via the following procedure:
%\begin{enumerate}
%  \item Go to the github organization for the course:
%        \url{https://github.com/NE204-Spring2018}
%  \item Create a new repository with the name <your\_github\_username>-lab0
%        (replace ``<your\_github\_username>'' with your actual user name).
%  \item Add 

\end{document}
