\documentclass[11pt]{article}

\usepackage{amsmath}    % Expanded math
\usepackage{amssymb}    % Expanded math symbols
\usepackage{graphicx}   % For images
\usepackage[version=3]{mhchem}  % Nuclide formatting
\usepackage{hyperref}           % Hyperlinks to URLs

\usepackage{geometry}
\geometry{left=2.5cm,right=2.5cm,top=2.5cm,bottom=2.5cm}

% Metadata
\title{%
       NE 204: Advanced Concepts in Radiation Detection and \\ Measurement \\
       \Large\bf Experiment 0: Energy Calibration with Reproducible Workflows}
       \author{Ross Barnowski \\ {\tt rossbar@berkeley.edu}}
\date{Revised \today}

%==============================================================================
\begin{document}
\maketitle

\section*{Purpose}

The purpose of this ``experiment'' is to provide the opportunity to practice 
generating lab reports using the reproducible workflow required for this
course. 
Energy calibration of a HPGe detector (a ubiquitous task in radiation detection
laboratories) will be used as an example for practicing the workflow.
You must implement an approach to downloading data (see next section), 
extracting the spectral information, and using it to determine an energy 
calibration. 
The energy calibration will then be evaluated using (at least) one of the data
sets that was not used in the calibration procedure.
These results, along with any relevant discussion, are to be included in the
lab writeup.

\section*{Approach}

This is not a real lab in the sense that you will not be collecting data 
yourself.
The purpose is to practice writing lab reports, so data for the task of
energy calibration will be provided to you.
Of course, students are welcome to collecting their own callibration spectra
from any of the available HPGe detectors, but the emphasis must be on 
fulfilling the software-related tasks discussed below.

\subsubsection*{Data}

The data used for this experiment has already been collected.
It consists of pulse height spectra taken with a coaxial HPGe detector using 
5 different radionuclide calibration sources: \ce{^{241}Am}, \ce{^{133}Ba},
\ce{^{137}Cs}, \ce{^{60}Co}, and \ce{^{152}Eu}.
The MCA with which the data was collected had 13-bit resolution, yielding 8192
channels.
The data for all five pulse height spectra are stored in a single text file in
ASCII format. 
The first row of the file is a header indicating which radionuclide the 
data in the columns correspond to, followed by 8192 rows indicating the number
of counts measured in each MCA channel.
The link to the data file and the corresponding md5 checksum are available at:
\begin{itemize}
  \item {\bf Data:} \url{https://www.dropbox.com/s/k692avun0144n90/lab0_spectral_data.txt?dl=0}
  \item {\bf Checksum:} \url{https://www.dropbox.com/s/6jquiryg6jskii0/lab0_spectral_data.md5?dl=0}
\end{itemize}

{\bf Required}
\begin{itemize}
  \item Implement the "data" command in the Makefile such that the data at the 
        above link will be downloaded when the user types "make data". 
        \textit{Hint: take a look at "wget" or "curl" - command line tools 
                used for downloading files}
\end{itemize}

{\bf Optional}
\begin{itemize}
  \item Implement a "validate" command in the Makefile that will compute the
        md5 checksum for the downloaded data, and compare it against the 
        provided checksum to verify that the data has not been corrupted.
        \textit{Hint: the md5sum command with the -c flag might prove useful.}
\end{itemize}

\section*{Report}

\end{document}
