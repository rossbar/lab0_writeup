\documentclass[11pt]{article}

\usepackage{amsmath}    % Expanded math
\usepackage{amssymb}    % Expanded math symbols
\usepackage{graphicx}   % For images
\usepackage[version=3]{mhchem}  % Nuclide formatting
\usepackage{hyperref}           % Hyperlinks to URLs

\usepackage{geometry}
\geometry{left=2.5cm,right=2.5cm,top=2.5cm,bottom=2.5cm}

% Metadata
\title{%
       NE 204: Advanced Concepts in Radiation Detection and \\ Measurement \\
       \Large\bf Experiment 0: Energy Calibration with Reproducible Workflows}
       \author{Ross Barnowski \\ {\tt rossbar@berkeley.edu}}
\date{Revised \today}

%==============================================================================
\begin{document}
\maketitle

\section*{Purpose}

The purpose of this ``experiment'' is to provide the opportunity to practice 
generating lab reports using the reproducible workflow required for this
course. 
Energy calibration of a HPGe detector (a ubiquitous task in radiation detection
laboratories) will be used as an example for practicing the workflow.
You must implement an approach to downloading data (see next section), 
extracting the spectral information, and using it to determine an energy 
calibration. 
The energy calibration will then be evaluated using (at least) one of the data
sets that was not used in the calibration procedure.
These results, along with any relevant discussion, are to be included in the
lab writeup.

\section*{Approach}

\section*{Report}

\end{document}
